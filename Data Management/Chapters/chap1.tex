\section{Introduction}
\setcounter{page}{1}
\pagenumbering{arabic}

The course is based on the following topics:
\begin{itemize}
    \item \textbf{The structure of a Data Base Management System (DBMS)}: Realtional data and queries, Buffer manager;
    \item \textbf{Transaction management}: The concept of transaction, Concurrency management;
    \item \textbf{Crash management}: Classification of failures, Recovery;
    \item \textbf{Data Warehousing}: Data warehousing architectures and operators, Data warehousing design;
    \item \textbf{NoSQL databases}: Document-based databases (such as MongoDB), Graph databases OLAP vs OLTP (such as Neo4j);
    \item \textbf{Physical structures for data bases}: File organizations for data base management, Principles of physical database design;
    \item \textbf{Query processing}: Evaluation of realational algebra operatos, Fundamentals of query optimization;
\end{itemize}

\subsection{The relational data model}
A \textbf{database} in the Realtional Model is a \textbf{set of tables} (or \textbf{relations}).
Each \textbf{table} is a \textbf{set of rows} (or \textbf{tuples}). Each one with the \textbf{same set of columns} (or \textbf{attributes}).

\vspace{2em}
\begin{table}[ht!]
    \parbox{.30\linewidth}{
        \centering
        \[
        \renewcommand\arraystretch{1.2}
        \begin{array}{|c|c|c|}
            \hline
            A & \tikzmark{startup}B & C\tikzmark{endup} \\
            \hline
            &  &  \\ 
            \hline
            &  \color{red}{v} &  \\ 
        \end{array}
        \]

        \captionsetup{labelformat=empty}
        \caption*{$T_1$}
    }
    \begin{tikzpicture}[remember picture,overlay]
        \foreach \Val in {up}
        {
        \draw[rounded corners,red,thick]
        ([shift={(-0.5\tabcolsep,-0.5ex)}]pic cs:start\Val) 
            rectangle 
        ([shift={(0.5\tabcolsep,2ex)}]pic cs:end\Val);
        }
    \end{tikzpicture}
    \hfill
    \parbox{.30\linewidth}{
        \centering
        \begin{tabular}{ |c|c|c|c| } 
            \hline
                A & B & C \\
            \hline
                &  &  \\ 
            \hline
                &  &  \\ 
        \end{tabular}
        \captionsetup{labelformat=empty}
        \caption{$T_2$}
    }
    \dots
    \parbox{.30\linewidth}{
        \centering
        \begin{tabular}{ |c|c|c|c| } 
            \hline
                A & B & C \\
            \hline
                &  &  \\ 
            \hline
                &  &  \\ 
        \end{tabular}
        \captionsetup{labelformat=empty}
        \caption*{$T_n$}
    }
\end{table}
\vspace{2em}
{\it\color{red}v} is the value of the corresponding column and row. The attributes B and C form a \textbf{superkey}.

\vspace{1em}
We have then:
\begin{itemize}
    \item \textbf{Integrity constraint}: a rule at the level of the schema that all the rows must respect;
    \item \textbf{Superkey}: there cannot exist two or more rows that have the same value as the combination of multiple attributes;
    \item \textbf{Key}: attribute in a table;
    \item \textbf{Foreign key}: attributes in a table are a reference of another table;
    \item \textbf{Primary key}: special key that doesn't allow null values (a null value is a a special value that says that the value is missing).
\end{itemize}

\vspace{2em}
\begin{table}[ht!]
    \parbox{.45\linewidth}{
        \centering
        \[
        \renewcommand\arraystretch{1.2}
        \begin{array}{|c|c|c|}
            \hline
                A & B & C\\
            \hline
                & & c_1 \\ 
            \hline
                & & c_2 \\ 
            \hline
                & & c_3 \\ 
            \hline
                & & c_3 \\ 
        \end{array}
        \]

        \captionsetup{labelformat=empty}
        \caption*{$T_1$ {\color{red} Unordered set}}
    }
    \parbox{.45\linewidth}{
        \centering
        \[
        \renewcommand\arraystretch{1.2}
        \begin{array}{|c|c|c|}
            \hline
                D & E & F\\
            \hline
                c_0 & & \\ 
            \hline
                c_1 & & \\ 
            \hline
                c_2 & & \\ 
            \hline
                c_3 & & \\ 
        \end{array}
        \]

        \captionsetup{labelformat=empty}
        \caption*{$T_2$ {\color{red} Must not miss any key}}
    }
\end{table}

\vspace{2em}

\begin{table}[ht!]
    \parbox{.45\linewidth}{
        \centering
        \[
        \renewcommand\arraystretch{1.2}
        \begin{array}{|c|c|c|}
            \hline
                A & B & C\\
            \hline
                a_1 & b_1 & c_1 \\ 
            \hline
                a_1 & b_2 & c_2 \\ 
            \hline
                a_2 & b_2 & c_3 \\ 
            \hline
               null & b_2 & c_3 \\ 
        \end{array}
        \]

        \captionsetup{labelformat=empty}
        \caption*{$T_1$ {\color{red} Unordered set}}
    }
    \parbox{.45\linewidth}{
        \centering
        \[
        \renewcommand\arraystretch{1.2}
        \begin{array}{|c|c|c|}
            \hline
                D & E & F\\
            \hline
                c_0 & & \\ 
            \hline
                c_1 & & \\ 
            \hline
                c_2 & & \\ 
            \hline
                c_3 & & \\ 
        \end{array}
        \]

        \captionsetup{labelformat=empty}
        \caption*{$T_2$ {\color{red} Must not miss any key}}
    }
\end{table}
We have a "no predicate" on the null value: never equal and never different, comparation is always false.
\vspace{2em}

As we have said, the Relational Data Model uses the mathematical concept of a relation as the formalism for describing and representing data.
A relation is a subset of a cartesian product of sets.
A relation can be considered as a "table" with rows and columns.

Codd introduced two different query languages for the relational data model:
\begin{itemize}
    \item \textbf{Relational Algebra}, which is a procedural language. It is an algebraic formalism in which queries are expressed by applying a sequence of operations to relations.
    \item \textbf{Relational Calculus}, which is a declarative language. It is a logical formalism in which queries are expressed as formulas of fist-order logic.
\end{itemize}

\textbf{Codd's Theorem}: Relational Algebra and Relational Calculus are essentially equivalent in terms of expressive power.

DBMSs are based on \textbf{SQL}, a hybrid of a procedural and a declarative language that combines features from both relational algebra and relational calculus.

\subsection{Relational algebra}
